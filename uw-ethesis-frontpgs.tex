% T I T L E   P A G E
% -------------------
% Last updated June 14, 2017, by Stephen Carr, IST-Client Services
% The title page is counted as page `i' but we need to suppress the
% page number. Also, we don't want any headers or footers.
\pagestyle{empty}
\pagenumbering{roman}

% The contents of the title page are specified in the "titlepage"
% environment.
\begin{titlepage}
        \begin{center}
        \vspace*{1.0cm}

        \Huge
        {\bf Micromeda: a genome property prediction pipeline and web visualization tool}

        \vspace*{1.0cm}

        \normalsize
        by \\

        \vspace*{1.0cm}

        \Large
        Lee Bergstrand \\

        \vspace*{3.0cm}

        \normalsize
        A thesis \\
        presented to the University of Waterloo \\ 
        in fulfillment of the \\
        thesis requirement for the degree of \\
        Master of Science \\
        in \\
        Biology \\

        \vspace*{2.0cm}

        Waterloo, Ontario, Canada, 2020 \\

        \vspace*{1.0cm}

        \copyright\ Lee Bergstrand 2020 \\
        \end{center}
\end{titlepage}

% The rest of the front pages should contain no headers and be numbered using Roman numerals starting with `ii'
\pagestyle{plain}
\setcounter{page}{2}

\cleardoublepage % Ends the current page and causes all figures and tables that have so far appeared in the input to be printed.
% In a two-sided printing style, it also makes the next page a right-hand (odd-numbered) page, producing a blank page if necessary.

% D E C L A R A T I O N   P A G E
% -------------------------------
  % The following is a sample Delaration Page as provided by the GSO
  % December 13th, 2006.  It is designed for an electronic thesis.

\begin{center}\textbf{Author's Declaration}\end{center}  
  
  
  \noindent
I hereby declare that I am the sole author of this thesis. This is a true copy of the thesis, including any required final revisions, as accepted by my examiners.

  \bigskip
  
  \noindent
I understand that my thesis may be made electronically available to the public.

\cleardoublepage

% A B S T R A C T
% ---------------

\begin{center}\textbf{Abstract}\end{center}

Understanding the distribution of biochemical pathways across microorganisms is 
critical to understanding these organism's evolution, ecology, and industrial 
applicability. Advances in genome sequencing and pathway databases have made 
genomically predicting what pathways an organism possesses a common technique. 
Researchers are moving on to scaling such analyses towards comparing the 
presence and absence of pathways across multiple microbes from the same 
environment or lineage. However, performing such analyses at scale is currently 
bottlenecked by the sheer number of pathways per organism and the lack of 
powerful tools to facilitate such comparisons. 

This thesis presents a new set of tools, called Micromeda, that will assist 
users in making comparative genomic analyses. Micromeda consists of three core 
components. These components are Micromeda-Client, which generates interactive 
heat maps that allow users to perform visual pathway comparisons; 
Micromeda-Server, which provides data to Micromeda-Client; and Pygenprop, which 
allows users to perform programmatic comparisons of multiple organism pathways. 
Micromeda uses the Genome Properties database as its pathway information source. 
This database is unique from other pathway databases because it maps directly 
between protein domains and pathway steps. The domains that the database uses 
are those from the InterPro consortium of protein databases. 

With Micromeda, the process of discovering an organism's pathways begins with 
the domain annotation of an organism's proteins by InterProScan. Afterwards, 
Pygenprop is used to combine these annotations with information from the Genome 
Properties database to predict biochemical pathways. This prediction of pathways 
from domain data results in the creation of a Micromeda file. This novel file 
type carries both the pathway annotations for multiple organisms and the 
sequences of proteins that support these annotations. In the context of the 
Genome Properties database, such pathways are referred to as genome properties, 
and pathway annotations are referred to as property assignments. The newly 
created Micromeda file can later be uploaded to Micromeda-Client and Server for 
heat map-based visualization.

Pygenprop uses object orient programming techniques to represent the Genome 
Properties database as a series of in-memory objects. These objects are used 
extensively within Pygenprop's property assignment process and Micromeda as a 
whole. Pygenprop is written in Python. The library's tight integration with 
the Python data science ecosystem, which results in it being compatible with 
many emerging data science and machine learning tools, lays the foundation for 
the library becoming the backbone of a new generation of automated pathway 
analysis tools.

Micromeda-Server is a Python web server application that provides data from 
uploaded Micromeda files to Micromeda-Client. Micromeda-Server makes data 
accessible via a web application programming interface (API). The API provides 
clients, such as Micromeda-Client, with access to property assignments and 
protein sequences found within uploaded Micromeda files. The API can also 
provide information about individual pathways and the overall structure of the 
Genome Properties database.

Micromeda-Client is a web client application whose purpose is to provide 
interactive pathway analysis heat maps to users. These heat maps are used to 
compare pathways across organisms within a dataset. The interactivity of these 
heat maps allows for pathway annotations to be aggregated into summaries of 
multiple pathways or be disaggregated down to a pathway step level. At a step 
level, users can see differences in the presence of pathways steps. Individual 
pathways of interest can also be looked up via text search. The heat map 
interface also allows users to download protein sequences that support 
individual pathway steps across multiple organisms.

Rather than having to spend time reviewing spreadsheets of pathway annotations 
or using existing ineffectual pathway annotation visualization software, 
researchers can now perform their analyses using Micromeda's streamlined and 
efficient heat maps. For large datasets, Pygenprop can be used to compare the 
predicted pathways of multiple organisms programmatically. Micromeda has the 
potential for shaping the way that future researchers perform pathway analysis.


\cleardoublepage

% A C K N O W L E D G E M E N T S
% -------------------------------

\begin{center}\textbf{Acknowledgements}\end{center}

I want to thank my supervisors, Dr. Josh Neufeld and Dr. Andrew Doxey, for their 
constant support and mentorship throughout the project. Feedback from my thesis 
committee, Dr. Laura Hug and Dr. Jim Wallace, was also much appreciated. I would 
also like to thank Jackson Tsuji and Dr. Sofie Thijs for their advice and 
feedback on user interface design. Concerning funding, I want to thank the 
Province of Ontario and the University of Waterloo for awarding me with a Queen 
Elizabeth II Graduate Scholarship in Science and Technology (QEII-GSST). I would 
also like to thank the Nuclear Waste Management Organization (NWMO) and Natural 
Sciences and Engineering Research Council (NSERC) for providing funding via a 
Collaborative Research and Development (CRD) grant awarded to my supervisors. 
Finally, I would like to thank Donya Saghattchi and John Zubak, from my startup 
company Amplytica Inc., for their consistent patience and support throughout the 
thesis process.

Figures \ref{fig:pathway-analysis-overview}, \ref{fig:micromeda-levels}, 
\ref{fig:micromeda-file-generation}, and 
\ref{fig:micromeda-file-building-and-use} contain icons made by Freepik 
(\href{http://flaticon.com/authors/freepik}{flaticon.com/authors/ freepik}), 
Smashicons 
(\href{http://flaticon.com/authors/smashicons}{flaticon.com/authors/smashicons}), 
icongeek26 (\href{http://flaticon.com/authors/icongeek26}{flaticon.com/authors/ 
icongeek26}), and Gregor Cresnar 
(\href{http://flaticon.com/authors/gregor-cresnar}{flaticon.com/authors/gregor-cresnar}). 
The icons were acquired from \href{http://flaticon.com}{flaticon.com} under 
Flaticon Basic License (see \href{http://file000.flaticon.com/downloads/license/ 
license.pdf}{file000.flaticon.com/downloads/ license/license.pdf}).

\cleardoublepage

% T A B L E   O F   C O N T E N T S
% ---------------------------------
\renewcommand\contentsname{Table of Contents}
\tableofcontents
\cleardoublepage
\phantomsection    % allows hyperref to link to the correct page

% L I S T   O F   F I G U R E S
% -----------------------------
\addcontentsline{toc}{chapter}{List of Figures}
\listoffigures
\cleardoublepage
\phantomsection		% allows hyperref to link to the correct page

% L I S T   O F   T A B L E S
% ---------------------------
\addcontentsline{toc}{chapter}{List of Tables}
\listoftables
\cleardoublepage
\phantomsection		% allows hyperref to link to the correct page

% GLOSSARIES (Lists of definitions, abbreviations, symbols, etc. provided by the glossaries-extra package)
% -----------------------------
\printglossaries
\cleardoublepage
\phantomsection		% allows hyperref to link to the correct page

% Change page numbering back to Arabic numerals
\pagenumbering{arabic}

