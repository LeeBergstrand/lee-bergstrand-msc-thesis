\chapter{Conclusions} \label{conclusion-chapter}

Micromeda is designed to help researchers compare the genomically predicted 
functional characteristics of multiple organisms. Two key elements facilitate 
this capability. Firstly, Pygenprop, a library that assists Micromeda's other 
components, can be used separately to compare functional characteristics across 
organisms programmatically. Secondly, Micromeda-Client can be used to generate 
an interactive heat map that allows users to perform the same comparisons 
visually. Micromeda-Server was developed to provide an \gls{api} that provides 
data to this client. Micromeda files allow for the transfer of entire datasets 
of pathway predictions and supporting information such as protein sequences to 
Micromeda-Client. They also enable the rapid transfer of datasets between 
researchers. Pygenprop is used to generate these Micromeda files, and the client 
uses their contents to draw its heat maps. The functional predictions made and 
displayed by Micromeda are derived from the information found within the Genome 
Properties database and the domain annotations of an organism's proteins. 

\section{Micromeda in the Context of Previous Work}

In contrast to many pathway annotation systems (see Sections 
\ref{server-summary} and \ref{visualization-comparison}) that are either run 
remotely or locally on a researcher's computer, Micromeda has both local and 
remote components. InterProScan and Pygenprop run locally on users' computers 
and generate Micromeda files. These files are later uploaded to Micromed-Client, 
which is deployed on a remote server along with Micromeda-Server. The main 
reason for this split is to shift the computational cost of generating pathway 
annotations onto users while allowing \gls{ui} components, such as 
Micromeda-Client, to be centrally hosted. This central hosting will allow for 
the rapid deployment of \gls{ui} updates and bug fixes.

From a human-computer interaction perspective, Micromeda provides a 
much-improved \gls{ui} for visualizing the difference in predicted function 
characteristics (Section \ref{visualization-comparison}). Unlike other tools, 
Micromeda-Client uses interactivity to dynamically switch between displaying 
summaries of the presence of multiple pathways and the presence of individual 
pathways steps. The client also integrates various pieces supporting information, such 
as descriptions of select genome properties, directly into its interface. This 
level of integration is in contrast to other tools that require users to view 
such information in separate web pages.

Another key feature of the Micromeda that makes it stand out from other 
tools is the platform's ability to connect pathway annotations to the predict protein 
sequences that underpin them. Both Pygenprop and Micromeda-Client allow users to 
generate \gls{fasta} files that contain proteins, from all organisms in a dataset, 
that support a property step. These \gls{fasta} files could be used to build 
phylogenetic trees. This feature is not available in other tools.

Pygenprop is one of the first libraries designed to support the programmatic 
comparison of genomically predicted functional characteristics. When combined 
with tools such as Jupyter Notebooks, the library provides a powerful tool for 
both developing pathway analysis tools and performing rapid analyses. Pygenprop 
is the second library that is compatible with Genome Properties data and is one 
of the only pathway comparison libraries written Python. The library is also one 
of the first pathway analysis tools to integrate with the Python data science 
ecosystem.

\section{Why Micromeda is an Important Tool That Will Support Future Research}

As the number of sequenced genomes increases and the cost of sequencing becomes 
more affordable, there will be an increasing need for tools that can rapidly 
compare the pathways of multiple organisms or even entire genome-resolved 
microbial communities. Micromeda provides tools that can scale with these 
increasing datasets. Micromeda-client can be used to compare the possessed 
pathways tens of organisms, and Pygenprop can examine the possessed pathways of 
thousands. Micromeda's improved user interface design and ability to connect 
pathway annotations to protein sequences will improve the velocity at which 
researchers can perform pathway analysis.

\section{Future Improvements to Micromeda}

As discussed in Section \ref{introduction_summary}, the summary of each chapter 
of the thesis examines potential improvements to the component that the chapter 
reviews. However, some substantial improvements require modifications to 
multiple parts of the Micromeda platform or are outside the scope of previous 
chapters. These changes are discussed in the subsections below.

\subsection{Creation of an Automated Pipeline for Rapid and Easy Generation of 
Micromeda Files} \label{pipeline-development}

Currently, there is no automated pipeline that users can apply to generate 
Micromeda files. As detailed in Section \ref{micromeda-overview}, users are 
required to install and run three \gls{cli} bioinformatics tools on their 
organisms' raw genome sequences to build Micromeda files. For convenience, an 
automated bioinformatics pipeline should be developed that would first install 
all these bioinformatics tools and later run them in series on user-supplied 
genomes. Components of the pipeline could be deployed via Conda 
(\href{http://conda.io}{conda.io}) and run in Conda environments to prevent 
these components from interfering with software installed previously. Pipeline 
automation tools such as Snakemake \cite{koster2012snakemake} or Nextflow 
\cite{di2017nextflow} could be used to develop such a pipeline. These tools 
ensure that steps in the pipeline are executed in the correct order and allow 
for these steps to be scaled out in parallel if multiple organism's genomes are 
to be processed.

An extension to this primary pipeline could also be used to speed up the process 
of generating Micromeda files. As discussed in Subsection 
\ref{why-micromeda-files}, InterProScan is very slow because it must scan an 
organism's predicted proteins with many domain models. One way of speeding up 
this process would be to reduce the set of proteins that are required to be 
scanned. In Subsection \ref{Genome-Properties-Files}, it is mentioned that each 
property in the Genome Properties database posses has an associated \gls{fasta} file 
containing representative protein sequences that carry out the steps of a 
property. A fast sequence similarity search tool such as \gls{mmseqs2} 
\cite{steinegger2017mmseqs2} could be used to compare an organism's proteins to 
the representatives in these \gls{fasta} files. This comparison could be used to 
filter out proteins that are unlikely to be homologous to those representative 
proteins from the database (\textit{i}.\textit{e}., true negatives), which reduces the overall set 
proteins that would need to be scanned by InterProScan. The Micromeda file 
generation pipeline mentioned above could automate these filtering steps.

\subsection{Add the Option to Assign Properties According Percentage 
Completeness Rather Than Categorically.}

Currently, the heat map generated by Micromeda is coloured according to property 
assignments that follow a discrete scale. Each property is assigned YES, 
PARTIAL, or NO support and is tinted to match. It should be noted that the 
algorithm employed to generate these assignments uses only the presence of 
required steps for a property in its calculations (see Subsection 
\ref{AssignmentCachingAlgorithm}). If even a single required step is not 
supported, then the parent property is assigned PARTIAL support. If Micromeda is 
applied to incomplete genomes, such as those generated from Metagenomes, then 
many genes encoding for proteins that support property steps may be missing. 
These missing genes would result in many properties being assigned PARTIAL 
support. 

Users may find it useful to be able to quantify the level of support for each 
assigned property along a continuous scale. This change would address these 
potentially high levels of PARTIAL property assignments created by running 
Micromeda on incomplete genomes. For example, properties that have more of their 
steps supported could be given higher values than those that are incomplete. 
Non-required steps could also influence such a continuous scale. In the context 
of a heat map, these continuously scaled assignments of support would be 
assigned different shades of the same colour.

Algorithms for calculating levels of PARTIAL support for biochemical pathways 
have been developed before. For example, \gls{maple} \cite{takami2016automated}, 
a \gls{kegg}-based pathway annotation server, can calculate \gls{mcr} (\textit{i}.\textit{e}., 
continuous levels of support based on the presence of pathway steps) for each 
module in the \gls{kegg}  database. \gls{kegg} modules are roughly equivalent to 
higher level genome properties. \gls{mcr}'s are calculations of the level of 
completeness of individual pathways that take into account that multiple 
different enzymes may catalyze some pathway steps. They are calculated using 
custom ``boolean algebra-like equations" generated for each \gls{kegg} module 
\cite{takami2012evaluation}. These equations take in \gls{ko} annotations (see 
Section \ref{visualization-comparison}, Fig. \ref{fig:functree2}, and 
\cite{mao2005automated}) of an organism's proteins as input. \gls{ko} 
annotations are equivalent to the step evidence used to support genome property 
steps. 

Pygenprop could also implement a similar ``completeness ratio" algorithm to that 
used by \gls{maple}. Much of the logic encoded in \gls{maple}'s ``boolean 
algebra-like equations" is already built into the Genome Properties database 
itself. For example, the record for each genome properties already records the 
InterPro domains, which represent multiple enzyme families, that can be used to 
support a single step. The steps required for individual properties are also 
codified. Calculation of ``completeness ratios" for genome properties may be 
complicated by the fact that several genome properties rely on the presence of 
others (see Section \ref{genome-properties-overview}), unlike \gls{kegg} 
modules, which are only supported by \gls{ko} annotations.

Pygenprop could implement such a ``completeness ratio" based assignment system 
by providing the library with a separate set of functions for calculating genome property 
assignments. The library could then be modified to calculate either discrete or 
continuous assignments for individual properties. These continuous assignments 
could be stored in Micromeda files or generated dynamically. Micromeda's 
interface could be modified to have a user interface switch that would allow 
users to switch their current heat map between displaying property assignments 
as either YES, NO, or PARTIAL or as a percentage of step completeness. 
Micromeda-Server would have to be modified to provide an endpoint (Section 
\ref{endpoints}) for these serving completeness-based property assignments to 
Micromeda-Client.

\subsection{Expansion of the Genome Properties Database to Included New 
Properties Not In InterPro}

Although the Genome Properties database has been shown to have similar overall 
genome coverage to other pathway databases (see Subsection 
\ref{reason-for-genome-properties-selection}), it still missing several pathways 
that are covered by other databases and pathways that are newly emerging in the 
literature. These pathways could be rapidly added to the Genome Properties 
database by adding a series of \gls{hmm}s \cite{eddy2011accelerated} to 
InterProScan's data files. These \gls{hmm}s would be used to detect the enzymes 
that carry out these new pathway's steps. For example, the \gls{fungene} 
\cite{fish2013fungene} database contains a series of custom HMMER 
\cite{eddy2011accelerated} \gls{hmm}s that are used for identifying enzymes 
involved in biogeochemical processes. Integrating \gls{fungene} as a new InterProScan 
member database would only require adding \gls{fungene}'s \gls{hmm}s to InterProScan's 
data files, writing a Python script for filtering out false-positive \gls{hmm} 
matches, and modifying some of InterProScan's configuration files. Afterwards, 
InterProScan would automatically use these new \gls{hmm}s when annotating novel 
proteins and would add the filtered results from running these \gls{hmm}s to its 
output \gls{tsv} file (see Section \ref{overview-interproscan}). This modified 
version of InterProScan could then be maintained separately or merged into the 
central InterProScan GitHub repository. Once \gls{fungene} is added to InterProScan, 
new Genome Properties could be created that use \gls{fungene} \gls{hmm}s names as step 
evidence instead of InterPro signature accessions. These novel properties could 
be implemented by added new property \textbf{DESC} files (Subsection 
\ref{Genome-Properties-Files}) to the database. This custom version of the 
Genome Properties database could also be maintained as a fork or merged into the 
latest version of the Genome Properties database on Github. A similar design 
pattern could be used to integrate any \gls{hmm}-based pathway database. The 
main advantage of this approach is that it allows the new database's \gls{hmm}s 
to be run by InterProScan, which provides high scalability across many \gls{cpu} 
cores or even on \gls{hpc} clusters.

\section{Recommendations for Future Development}

Researchers are scaling up their research projects from studying the 
capabilities of individual microorganisms to looking at the functional 
capabilities of entire microbial communities from a systems biology perspective. 
Due to its ability to programmatically compare thousands of pathway presence and 
absences from multiple organisms simultaneously, Pygenprop has the potential to 
form the basis of future bioinformatics tools that support the above emerging 
community-scale research. Such tools could use patterns in the presence and 
absence of pathway steps found in organisms from a single environment to detect 
patterns of interspecies cross-feeding. Pygenprop's integration with the Python 
data science ecosystem could be leveraged to build classifiers that 
automatically identify organisms that carry out specific ecosystem functions. 
When combined with culture condition information found in databases such as 
\gls{bacdive} \cite{reimer2018bac}, Pygenprop could be used to develop classifiers 
that use the presence and absence of genome properties to predict an organism's 
favoured growth conditions. If such a tool was applied to multiple organisms 
from an environment, the data produced could be potentially used to build tools 
that predict how microbial communities will shift in response to changes in 
environmental conditions. It is recommended that future users of Pygenprop 
explore the possibility of building such tools.

\section{Summary}

Micromeda is a set of tools that allow users to make rapid comparisons of the 
pathways possessed by multiple organisms. Although there are many improvements 
to be made to the software, the current version of the tool is robust and 
provides users will new capabilities with regards to pathway analysis. The 
platform should be maintained and further expanded upon in terms of both user 
interface and overall features. Micromeda already provides users with a faster 
way to perform pathway analysis and future improved versions, with improved 
performance and capabilities, will help users even more. It is hoped that 
Micromeda will garner wide adoption by the scientific community.
